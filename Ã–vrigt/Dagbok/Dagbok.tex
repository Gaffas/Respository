\documentclass[a4paper,oneside]{book}
\usepackage{lipsum}
\usepackage{xcolor}
\usepackage{framed}
\usepackage{datetime}
\usepackage[utf8]{inputenc}
\usepackage[T1]{fontenc}
\usepackage{fourier}
\usepackage{marginnote}
\usepackage{tikz}

\usepackage[top=4cm, bottom=2cm, outer=2cm, inner=5cm, heightrounded, marginparwidth=3.5cm, marginparsep=0.75cm]{geometry}

\newcommand{\mybox}[1]{%
\marginnote{{\color{red}CAUTION!}\\{#1}}
}

\newcommand{\vecka}[1]{%
\marginnote{{\color{blue}\huge{Vecka #1}}}
}

\newlength\sidebar
 \newlength\envrule
 \newlength\envborder
 \setlength\sidebar{1.5mm}
 \setlength\envrule{0.4pt}
 \setlength\envborder{2mm}

\makeatletter
 \long\def\fboxs#1{%
   \leavevmode
   \setbox\@tempboxa\hbox{%
     \color@begingroup
       \kern\fboxsep{#1}\kern\fboxsep
     \color@endgroup}%
   \@frames@x\relax}
 \def\frameboxs{%
   \@ifnextchar(%)
     \@framepicbox{\@ifnextchar[\@frameboxs\fboxs}}
 \def\@frameboxs[#1]{%
   \@ifnextchar[%]
     {\@iframeboxs[#1]}%
     {\@iframeboxs[#1][c]}}
 \long\def\@iframeboxs[#1][#2]#3{%
   \leavevmode
   \@begin@tempboxa\hbox{#3}%
     \setlength\@tempdima{#1}%
     \setbox\@tempboxa\hb@xt@\@tempdima
          {\kern\fboxsep\csname bm@#2\endcsname\kern\fboxsep}%
     \@frames@x{\kern-\fboxrule}%
   \@end@tempboxa}
 \def\@frames@x#1{%
   \@tempdima\fboxrule
   \advance\@tempdima\fboxsep
   \advance\@tempdima\dp\@tempboxa
   \hbox{%
     \lower\@tempdima\hbox{%
       \vbox{%
         %\hrule\@height\fboxrule
         \hbox{%
          \vrule\@width\fboxrule
           #1%
           \vbox{%
             \vskip\fboxsep
             \box\@tempboxa
             \vskip\fboxsep}%
           #1%
           }%\vrule\@width\fboxrule}%
         }%\hrule\@height\fboxrule}%
                           }%
         }%
 }
 \def\esefcolorbox#1#{\esecolor@fbox{#1}}
 \def\esecolor@fbox#1#2#3{%
   \color@b@x{\fboxsep\z@\color#1{#2}\fboxs}{\color#1{#3}}}
 \makeatother


 \definecolor{exampleborder}{HTML}{FE642E}
 \definecolor{examplebg}{HTML}{CEF6EC}
 \definecolor{statementborder}{rgb}{.9,0,0}
 \definecolor{statementbg}{rgb}{1,1,1}

 \newenvironment{eseframed}{%
   \def\FrameCommand{\fboxrule=\the\sidebar  \fboxsep=\the\envborder%
   \esefcolorbox{exampleborder}{examplebg}}%
   \MakeFramed{\FrameRestore}}%
  {\endMakeFramed}

%\renewcommand\dateTurkish{\def\today{\number\day~%
 %\ifcase \month \or Ocak\or Şubat\or Mart\or Nisan\or Mayıs\or Haziran\or
 %  Temmuz\or Ağustos\or Eylül\or Ekim\or Kasım\or Aralık\fi\space
 %\number\year}}
%\dateTurkish

 \newcounter{diary}
%\numberwithin{uygulama}
\renewcommand{\thediary}{\arabic{diary}}

 %%% CODE ENVIRONMENT. PUT TEXT INTO COLORED FRAME %%%
 \newenvironment{diary}[2]
 {\par\medskip\refstepcounter{diary}%
 \hbox{%
 \fboxsep=\the\sidebar\hspace{-\envborder}\hspace{-.5\sidebar}%
 \colorbox{exampleborder}{%
 \hspace{\envborder}\footnotesize\sffamily\bfseries%
 \textcolor{white}{{#1}\ {#2}\enspace\hspace{\envborder}}
%\today
 }
 }
 \nointerlineskip\vspace{-\topsep}%
 \begin{eseframed}\noindent\ignorespaces%
 }
 {\end{eseframed}\vspace{-\baselineskip}\medskip}

\title{\Huge Dagbok}
\author{Nicklas Hasselström}
\date{}
\begin{document}
\reversemarginpar
\pagenumbering{gobble}
\maketitle
\newpage

\begin{diary}{Måndag}{2016-02-01 08:00-17:30}
\vecka{5}
	Idag kom jag till kontoret för första gången. Jag hade fått en dator och jag kopplade in och ställde allt till rätta. Efter det så var det stand-up-möte. Teamet, bestående av Richard, Rickard, Mattias, Stefan och Lars gick igenom vad de hade gjort i fredags och vad de planerade göra under dagen.\\
	Fortsatte dagen med att konfigurera datorn och logga in på alla konton. Hade problem med att genomföra tvåstegsverifieringen för att komma åt gmail. Fick prata med tech-supporten för att få en verifieringsnyckel. Efter detta hade vi möte med siten i Norge. Jag fick vara med på detta möte med och lyssna.
\end{diary}

\begin{diary}{Tisdag}{2016-02-02 08:00-17:00}
%\mybox{Put some text here for caution.}
	Började dagen med att kontakta tech-support för att få local admin access till min arbetsstation. Tog ett tag då datorn är registrerad på någon Elisabeth. Dock så var det inga problem att få access då tech-supporten skötte  allt. Klockan nio hade teamet ett sprintplaneringsmöte på en timme. Jag var med på detta med och de lade till och med in mig i sprinten. Jag skall under de följande två veckorna planera mitt examensarbete och läsa in mig på Continua guidelines. Efter mötet fortsatte jag att installera olika verktyg, bl.a. \LaTeX\ och ett referenshanteringsprogram för att hantera mina .bib-filer jag kommer använda mig av vid rapportskrivandet. Vi hade ett diskussionsmöte för att ta fram underlag till en ny testutvecklingsmiljö som framtida testfall skall skrivas till. Lyckades äntligen ställa in mallarna för rapporten så nu är det bara att börja skriva. Alla konton jag behöver komma åt är nu konfigurerade.
\end{diary}

\begin{diary}{Onsdag}{2016-02-03 08:00-15:00}
	Idag var en kort dag. Jag gjorde lite finjusteringar i rapportmallen och hade sedan ett kort stand-up-möte, jag berättade om mina framsteg och vad jag skall göra under dagen. Klockan 10 var det dags för fotografering så jag kan få ett permanent ID-kort att ha på mig. Klockan 11:30 gick jag från jobbet för att hinna äta i skolan innan föreläsningen i informationssökning på universitetsbiblioteket startade 13:15. Jag lärde mig på föreläsningen hur man kan exportera referenser från olika databaser till \LaTeX .
\end{diary}

\begin{diary}{Torsdag}{2016-02-04 08:00-17:30}
	Började dagen med att skriva in dagboksinlägg från föregående dagar. Inget stand-up-möte idag. Bara jag och Richard på kontoret. Skrev färdigt specifikationen för examensarbetet samt satte upp ett git-repository för mina rapporter. Kommer inte lägga någon kod i mitt egna respository för att inte riskera sprida någon känslig information. Testar lite branchning i git för att komma in i rutinerna. Jag bestämde mig för att använda GitHub som verktyg för detta. Jag började läsa på om Continua i de guidelines jag laddade ned i tisdags. Otroligt mycket att gå igenom verkar det som. GitHub krånglade en hel del men till slut fick jag ordning på det. Går hem för dagen och planerar sätta upp en skrivarmiljö på datorn hemma. Känner att det inte kommer finnas tid att bara lägga 8 timmar om dagen på både rapporten och komma framåt med examensarbetet. Jag måste dess utom göra en riktig planering och börja uppskatta hur lång tid de olika delarna kommer ta. Se specifikationen för examensarbetet för att få en insikt i vad som skall göras.
\end{diary}

\begin{diary}{Fredag}{2016-02-05 08:00-17:00}
	Idag var jag på Compare-lunch som handlade om e-hälsa. Cecilia Karlsson från Landstinget Värmland och Maria Johansson från Karlstad Kommun var där och föreläste om möjligheter inom e-hälsa. Jag har idag även slutfört den grova planeringen för examensarbetet. På stand-up-mötet i morse berättade Stefan och Rickard om mässan de hade varit på igår. Mycket om säkerhet samt om hur viktigt det är med Continua certifiering för att slå på marknaden. Jag fick en ide om att man kanske kunde ta lösningen från S2SCTP-projektet för att få till en autentiseringslösning.
\end{diary}
\newpage

\begin{diary}{Måndag}{2016-02-08 08:00-16:00}
\vecka{6}
	Började idag med att läsa igenom rapporten som vi blev ålagda att läsa av Donald. Det tog ungefär tre timmar att läsa igenom den utan att fokusera på att hitta fel eller utföra opponeringsarbete. Installerade en svensk ordbok för Texmaker med så jag kan skriva dagboken felfritt. Blev även ett tech-support ärende idag då jag inte kommer åt proxy till kau via Tietos nät.\\
	Tech-support ärendet har fortfarande inte lösts nu när dagen är över. Kommer inte åt databaserna som biblioteket tillhandahåller. Verkar vara något problem med att komma åt utomstående proxyservrar från Tieto's nät. Har läst vidare om Continua. Det är svårt att säga något om continua och jag börjar bli osäker på om jag skall lägga ned allt för mycket tid på att utvärdera Continua eller om jag skall gå vidare med att läsa in om referenscases och börja designa en prototyp istället. Jag letade lite idag och hittade användarmanualen till Bluetooth med. Över 2000 sidor. Tänker inte läsa igenom den.\\
	Hittade ett open source projekt där upp till sju bluetooth-enheter kopplar till en server. Kan vara bra för mitt projekt. Det är dock skrivet i java vilket jag inte är så säker på. Kan ju ändra mig.
\end{diary}

\begin{diary}{Tisdag}{2016-02-09 08:00-16:00}
	Läste på vidare om Continua. Började störa mig på att det fortfarande inte går att komma åt databaserna via bibproxy, skolans proxyserver. Tech-support menade att det var skolans fel att jag inte kom åt deras proxy-server. Pratade med Rickard angående detta och fick tillgång till gästnätet. Detta gjorde att jag till slut kunde komma åt databaserna. Tyvärr så måste jag växla mellan att komma åt databaserna och komma åt intranätet genom att dra ur och sätta i ethernätsladden.\\
	På eftermiddagen var jag i skolan och lyssnade på Evry som höll i Snits-lunchen. De pratade om ett nytt system som de skall lansera senare i vår. Systemet är inom e-hälsa och verkar intressant för mitt examensarbete. Fick ett visitkort till Andreas Andersson som ville att jag skulle ringa. Kommer besöka deras monter på HotSpot mässan istället. Hade även handledarmöte idag för första gången och gick igenom examensarbetet med Thijs. Vi bestämde att han skulle komma på besök den tredje mars 15:00.
\end{diary}

\begin{diary}{Onsdag}{2016-02-10 08:00-16:30}
	Hade möte med Lars som kom tillbaka efter några dagars resande. Han gick igenom vad som var sagt på de mässor som han hade besökt den senaste veckan. Berättade om snits-lunchen och det var intressant även för resten av teamet. På eftermiddagen var det dags för sista föreläsningen fram till det är dags för oppositionen. Donald gick igenom vissa fallgropar som man kan råka ut för när man skriver rapporten. Blev en kort dag även idag då mycket av tiden gick åt att vänta.
\end{diary}

\begin{diary}{Torsdag}{2016-02-11 08:00-17:00}
	Började skriva ordentligt på rapporten idag. Avslutade inläsningen på continua i stora drag idag. 
\end{diary}

\begin{diary}{Fredag}{2016-02-12 08:00-16:00}
	Långsam dag, inte mycket gjort. Letade efter referenscase till continua och skrev på rapporten.
\end{diary}
\newpage

\begin{diary}{Måndag}{2016-02-15 08:00-16:00}
\vecka{7}
	Enbart rapportskrivning idag. Skrev bakrund och underrubriker inom Telehealth, Tieto, Lifecare och lite om mitt examensarbete.
\end{diary}

\begin{diary}{Tisdag}{2016-02-16 08:00-14:00}
	Börjat utvärdering om vilket OS som skall användas. Apple lägger mycket resuresr på att ta fram nya eHealth lösningar. Windows phone kör C\# samt har bra säkerhet, Android har BluetoothHealth vilket ser lovande ut. Diskuterade lite med Christian om detta på lunchen idag. Slutade tidigt för att hjälpa till med soffbärning. 
\end{diary}

\begin{diary}{Onsdag}{2016-02-17}
	Gick på HotSpot idag. Pratade med Nordic Medtest. De ser att fler och fler kommuner efterfrågar standarder. Kommer bli så att Continua skall bli implementerat i systemet. Pratade även med Xlent om deras x-jobbare som skall ta fram ett IoT system där bluetooth enheter skall kunna bilda nätverk.
\end{diary}

\begin{diary}{Torsdag}{2016-02-18 08:00-17:30}
	Börjar sätta upp en utvecklingsmiljö för Android. Installerade både Android Studio och Visual Studio med Xamarin. Visual studio kommer bli bra om jag bestämmer mig för att köra på stöd för flera OS. Android Studio passar för utveckling till Andriod. Kommer behöva skriva testfall med för att säkerhetsställa funktionaliteten i appen så har suttit och skrivit lite testfall och fått det att fungera. Läste även lite i ett dockument som beskriver tankarna hos de nordiska länderna gällande framtiden inom Personal Connected health and care.\\
	Proxyproblemet ställer fortfarande till det lite för mig. Det verkar som om jag måste vara online på intranätet om jag skall köra Visual Studio med Xamarin då Xamarins utvecklingsmiljö ligger på en server som man måste komma åt. Brandväggen blockar Xamarin Android Player om jag inte sitter på intranätet.
\end{diary}

\begin{diary}{Fredag}{2016-02-19 08:00-17:00}
	Började så smått skriva i utvecklingsmiljön för att få ett humm om hur det funkar med android. Bestämde mig för att det kommer bli android till slut. Fördelarna med en egen service för bluetooth gör att det blir bra.
\end{diary}
\newpage

\begin{diary}{Måndag}{2016-02-22 08:00-15:00}
\vecka{8}
	Började implementera ett skal för en android app, letade efter lösningar för att kunna testa bluetooth men det verkar som om jag måste ha en fysisk telefon. Fanns ett sätt att testa men då var jag tvungen att köra en brygga genom en viruell maskin som kör en androidenhets operativsystem. Lättare att testa fysiskt på en riktig enhet.
\end{diary}

\begin{diary}{Tisdag}{2016-02-23 08:00-16:00}
	Implementerade mycket i skalet för appen. Läste på om hur man programmerar android. Var mycket länge sedan jag gjorde det men det kommer tillbaka mer och mer. Kan hända att jag måste lägga mer tid på att lära mig android än vad jag först hade beräknat. Tiden vill ge svar på det. Skrev även lite på rapporten som jag börjar bli mer och mer stressad över. Kommer sitta med rapporten hemma ikväll verkar det som då jag kommer förlora en halv dag på torsdag i och med MSB.
\end{diary}





\end{document}
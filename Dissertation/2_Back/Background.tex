%Introduce problem area / give relevant background info
%Introduction - Explain WHY you are doing this study
%Information - Background / your study in the wider context
%Similar work (projects, systems etc.)
%Chapter Summary
The human species is getting older and older. Our life span is getting longer which means that we have more time to get sick. The longer we live, the more likely it will be that we get one of the chronic diseases. In 2010 there were 759 million people over the age of 60. This was at the time 11 percent of the total population. In 2050 this group is estimated to be more than 2 billion. Not only does the population increase, proportionally the population is getting older and in 2050 it is estimated that 22 percent will be at the age of 60 and older~\cite{UNpub}.\\
With fewer people taking care of a growing population, new methods needs to be implemented in order to cover the lack of personnel. One of these new methods is home care. If a patient is well enough to care for them self, then it would take a huge burden of the welfare system. A patient who has been living with for example diabetes, knows how to do the daily check up, he knows how to take blood values and know best the state of his illness. If he can live a rich life at home and care for himself he will be happier and it will will free time which can be better spent on patients who need more care.\\
However, due to the age and progression of the disease of a patient, the health centre might want to keep a closer look at him and do daily check ups. Today this is done by sending a nurse to the home of a patient or by making the patient come to the health centre for daily and weekly check ups. This takes time from both the patient and nurse. If the the patient could send in data regarding their physical status this would greatly reduce the time of health care issues. One method which regards these problem is the Telehealth technology.\\

\subsection{Telehealth}
\label{sub:telehealth}
Telehealth~\cite{telehealth} is a collection of medical equipment used to monitor and collect data of the patient in his or her home. Data which is collected can be from blood pressure, blood values, weight, surveillance of demented patients or security alarm.\\
%Skriv mer här

\subsection{Tieto}
\label{sub:tieto}

\subsubsection{Lifecare eSense}
\label{subsub:eSense}

\subsubsection{Mobile gateway}
\label{subsub:gateway}

\subsection{Continua}
\label{sub:continua}

\subsection{Chapter Summary}
\label{sub:backSum}
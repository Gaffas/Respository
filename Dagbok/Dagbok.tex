\documentclass[a4paper,oneside]{book}
\usepackage{lipsum}
\usepackage{xcolor}
\usepackage{framed}
\usepackage{datetime}
\usepackage[utf8]{inputenc}
\usepackage[T1]{fontenc}
\usepackage{fourier}
\usepackage{marginnote}
\usepackage{tikz}

\input{input}

\title{\Huge Dagbok}
\author{Nicklas Hasselström}
\date{}
\begin{document}
\reversemarginpar
\pagenumbering{gobble}
\maketitle
\newpage

\begin{diary}{Måndag}{2016-02-01 08:00-17:30}
\vecka{5}
	Idag kom jag till kontoret för första gången. Jag hade fått en dator och jag kopplade in och ställde allt till rätta. Efter det så var det stand-up-möte. Teamet, bestående av Richard, Rickard, Mattias, Stefan och Lars gick igenom vad de hade gjort i fredags och vad de planerade göra under dagen.\\
	Fortsatte dagen med att konfigurera datorn och logga in på alla konton. Hade problem med att genomföra tvåstegsverifieringen för att komma åt gmail. Fick prata med tech-supporten för att få en verifieringsnyckel. Efter detta hade vi möte med siten i norge. Jag fick vara med på detta möte med och lyssna.
\end{diary}

\begin{diary}{Tisdag}{2016-02-02 08:00-17:00}
%\mybox{Put some text here for caution.}
	Började dagen med att kontakta tech-support för att få local admin access till min arbetsstation. Tog ett tag då datorn är registrerad på någon Elisabeth. Dock så var det inga problem att få access då tech-supporten skötte  allt. Klockan nio hade teamet ett sprintplaneringsmöte på en timme. Jag var med på detta med och de lade till och med in mig i sprinten. Jag skall under de följande två veckorna planera mitt examensarbete och läsa in mig på Continua guidelines. Efter mötet fortsatte jag att installera olika verktyg, bl.a. \LaTeX\ och ett referenshanteringsprogram för att hantera mina .bib-filer jag kommer använda mig av vid rapportskrivandet. Vi hade ett diskussionsmöte för att ta fram underlag till en ny testutvecklingsmiljö som framtida testfall skall skrivas till. Lyckades äntligen ställa in mallarna för rapporten så nu är det bara att börja skriva. Alla konton jag behöver komma åt är nu konfigurerade.
\end{diary}

\begin{diary}{Onsdag}{2016-02-03 08:00-11:30}
	Idag var en kort dag. Jag gjorde lite finjusteringar i rapportmallen och hade sedan ett kort stand-up-möte, jag berättade om mina framsteg och vad jag skall göra under dagen. Klockan 10 var det dags för fotografering så jag kan få ett permanent ID-kort att ha på mig. Klockan 11:30 gick jag från jobbet för att hinna äta i skolan innan föreläsningen i informationssökling på universitetsbiblioteket startade 13:15. Jag lärde mig på föreläsningen hur man kan exportera referenser från olika databaser till \LaTeX .
\end{diary}

\begin{diary}{Torsdag}{2016-02-04 08:00-17:00}
	Började dagen med att skriva in daboksinlägg från föregående dagar. 
\end{diary}

\begin{diary}{Fredag}{2016-02-05 08:00-17:00}
\lipsum[1]
\end{diary}
\newpage

\begin{diary}{Måndag}{2016-02-08 08:00-17:00}
\vecka{6}
\lipsum[1]
\end{diary}






















\end{document}
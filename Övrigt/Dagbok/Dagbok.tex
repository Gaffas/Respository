\documentclass[a4paper,oneside]{book}
\usepackage{lipsum}
\usepackage{xcolor}
\usepackage{framed}
\usepackage{datetime}
\usepackage[utf8]{inputenc}
\usepackage[T1]{fontenc}
\usepackage{fourier}
\usepackage{marginnote}
\usepackage{tikz}

\input{input}

\title{\Huge Dagbok}
\author{Nicklas Hasselström}
\date{}
\begin{document}
\reversemarginpar
\pagenumbering{gobble}
\maketitle
\newpage

\begin{diary}{Måndag}{2016-02-01 08:00-17:30}
\vecka{5}
	Idag kom jag till kontoret för första gången. Jag hade fått en dator och jag kopplade in och ställde allt till rätta. Efter det så var det stand-up-möte. Teamet, bestående av Richard, Rickard, Mattias, Stefan och Lars gick igenom vad de hade gjort i fredags och vad de planerade göra under dagen.\\
	Fortsatte dagen med att konfigurera datorn och logga in på alla konton. Hade problem med att genomföra tvåstegsverifieringen för att komma åt gmail. Fick prata med tech-supporten för att få en verifieringsnyckel. Efter detta hade vi möte med siten i Norge. Jag fick vara med på detta möte med och lyssna.
\end{diary}

\begin{diary}{Tisdag}{2016-02-02 08:00-17:00}
%\mybox{Put some text here for caution.}
	Började dagen med att kontakta tech-support för att få local admin access till min arbetsstation. Tog ett tag då datorn är registrerad på någon Elisabeth. Dock så var det inga problem att få access då tech-supporten skötte  allt. Klockan nio hade teamet ett sprintplaneringsmöte på en timme. Jag var med på detta med och de lade till och med in mig i sprinten. Jag skall under de följande två veckorna planera mitt examensarbete och läsa in mig på Continua guidelines. Efter mötet fortsatte jag att installera olika verktyg, bl.a. \LaTeX\ och ett referenshanteringsprogram för att hantera mina .bib-filer jag kommer använda mig av vid rapportskrivandet. Vi hade ett diskussionsmöte för att ta fram underlag till en ny testutvecklingsmiljö som framtida testfall skall skrivas till. Lyckades äntligen ställa in mallarna för rapporten så nu är det bara att börja skriva. Alla konton jag behöver komma åt är nu konfigurerade.
\end{diary}

\begin{diary}{Onsdag}{2016-02-03 08:00-15:00}
	Idag var en kort dag. Jag gjorde lite finjusteringar i rapportmallen och hade sedan ett kort stand-up-möte, jag berättade om mina framsteg och vad jag skall göra under dagen. Klockan 10 var det dags för fotografering så jag kan få ett permanent ID-kort att ha på mig. Klockan 11:30 gick jag från jobbet för att hinna äta i skolan innan föreläsningen i informationssökning på universitetsbiblioteket startade 13:15. Jag lärde mig på föreläsningen hur man kan exportera referenser från olika databaser till \LaTeX .
\end{diary}

\begin{diary}{Torsdag}{2016-02-04 08:00-17:30}
	Började dagen med att skriva in dagboksinlägg från föregående dagar. Inget stand-up-möte idag. Bara jag och Richard på kontoret. Skrev färdigt specifikationen för examensarbetet samt satte upp ett git-repository för mina rapporter. Kommer inte lägga någon kod i mitt egna respository för att inte riskera sprida någon känslig information. Testar lite branchning i git för att komma in i rutinerna. Jag bestämde mig för att använda GitHub som verktyg för detta. Jag började läsa på om Continua i de guidelines jag laddade ned i tisdags. Otroligt mycket att gå igenom verkar det som. GitHub krånglade en hel del men till slut fick jag ordning på det. Går hem för dagen och planerar sätta upp en skrivarmiljö på datorn hemma. Känner att det inte kommer finnas tid att bara lägga 8 timmar om dagen på både rapporten och komma framåt med examensarbetet. Jag måste dess utom göra en riktig planering och börja uppskatta hur lång tid de olika delarna kommer ta. Se specifikationen för examensarbetet för att få en insikt i vad som skall göras.
\end{diary}

\begin{diary}{Fredag}{2016-02-05 08:00-17:00}
	Idag var jag på Compare-lunch som handlade om e-hälsa. Cecilia Karlsson från Landstinget Värmland och Maria Johansson från Karlstad Kommun var där och föreläste om möjligheter inom e-hälsa. Jag har idag även slutfört den grova planeringen för examensarbetet. På stand-up-mötet i morse berättade Stefan och Rickard om mässan de hade varit på igår. Mycket om säkerhet samt om hur viktigt det är med Continua certifiering för att slå på marknaden. Jag fick en ide om att man kanske kunde ta lösningen från S2SCTP-projektet för att få till en autentiseringslösning.
\end{diary}
\newpage

\begin{diary}{Måndag}{2016-02-08 08:00-16:00}
\vecka{6}
	Började idag med att läsa igenom rapporten som vi blev ålagda att läsa av Donald. Det tog ungefär tre timmar att läsa igenom den utan att fokusera på att hitta fel eller utföra opponeringsarbete. Installerade en svensk ordbok för Texmaker med så jag kan skriva dagboken felfritt. Blev även ett tech-support ärende idag då jag inte kommer åt proxy till kau via Tietos nät.\\
	Tech-support ärendet har fortfarande inte lösts nu när dagen är över. Kommer inte åt databaserna som biblioteket tillhandahåller. Verkar vara något problem med att komma åt utomstående proxyservrar från Tieto's nät. Har läst vidare om Continua. Det är svårt att säga något om continua och jag börjar bli osäker på om jag skall lägga ned allt för mycket tid på att utvärdera Continua eller om jag skall gå vidare med att läsa in om referenscases och börja designa en prototyp istället. Jag letade lite idag och hittade användarmanualen till Bluetooth med. Över 2000 sidor. Tänker inte läsa igenom den.\\
	Hittade ett open source projekt där upp till sju bluetooth-enheter kopplar till en server. Kan vara bra för mitt projekt. Det är dock skrivet i java vilket jag inte är så säker på. Kan ju ändra mig.
\end{diary}

\begin{diary}{Tisdag}{2016-02-09 08:00-16:00}
	Läste på vidare om Continua. Började störa mig på att det fortfarande inte går att komma åt databaserna via bibproxy, skolans proxyserver. Tech-support menade att det var skolans fel att jag inte kom åt deras proxy-server. Pratade med Rickard angående detta och fick tillgång till gästnätet. Detta gjorde att jag till slut kunde komma åt databaserna. Tyvärr så måste jag växla mellan att komma åt databaserna och komma åt intranätet genom att dra ur och sätta i ethernätsladden.\\
	På eftermiddagen var jag i skolan och lyssnade på Evry som höll i Snits-lunchen. De pratade om ett nytt system som de skall lansera senare i vår. Systemet är inom e-hälsa och verkar intressant för mitt examensarbete. Fick ett visitkort till Andreas Andersson som ville att jag skulle ringa. Kommer besöka deras monter på HotSpot mässan istället. Hade även handledarmöte idag för första gången och gick igenom examensarbetet med Thijs. Vi bestämde att han skulle komma på besök den tredje mars 15:00.
\end{diary}

\begin{diary}{Onsdag}{2016-02-10 08:00-16:30}
	Hade möte med Lars som kom tillbaka efter några dagars resande. Han gick igenom vad som var sagt på de mässor som han hade besökt den senaste veckan. Berättade om snits-lunchen och det var intressant även för resten av teamet. På eftermiddagen var det dags för sista föreläsningen fram till det är dags för oppositionen. Donald gick igenom vissa fallgropar som man kan råka ut för när man skriver rapporten. Blev en kort dag även idag då mycket av tiden gick åt att vänta.
\end{diary}

\begin{diary}{Torsdag}{2016-02-11 08:00-17:00}
	Började skriva ordentligt på rapporten idag. Avslutade inläsningen på continua i stora drag idag. 
\end{diary}

\begin{diary}{Fredag}{2016-02-12 08:00-16:00}
	Långsam dag, inte mycket gjort. Letade efter referenscase till continua och skrev på rapporten.
\end{diary}
\newpage

\begin{diary}{Måndag}{2016-02-15 08:00-16:00}
\vecka{7}
	Enbart rapportskrivning idag. Skrev bakrund och underrubriker inom Telehealth, Tieto, Lifecare och lite om mitt examensarbete.
\end{diary}

\begin{diary}{Tisdag}{2016-02-16 08:00-14:00}
	Börjat utvärdering om vilket OS som skall användas. Apple lägger mycket resuresr på att ta fram nya eHealth lösningar. Windows phone kör C\# samt har bra säkerhet, Android har BluetoothHealth vilket ser lovande ut. Diskuterade lite med Christian om detta på lunchen idag. Slutade tidigt för att hjälpa till med soffbärning. 
\end{diary}

\begin{diary}{Onsdag}{2016-02-17}
	Gick på HotSpot idag. Pratade med Nordic Medtest. De ser att fler och fler kommuner efterfrågar standarder. Kommer bli så att Continua skall bli implementerat i systemet. Pratade även med Xlent om deras x-jobbare som skall ta fram ett IoT system där bluetooth enheter skall kunna bilda nätverk.
\end{diary}

\begin{diary}{Torsdag}{2016-02-18 08:00-17:30}
	Börjar sätta upp en utvecklingsmiljö för Android. Installerade både Android Studio och Visual Studio med Xamarin. Visual studio kommer bli bra om jag bestämmer mig för att köra på stöd för flera OS. Android Studio passar för utveckling till Andriod. Kommer behöva skriva testfall med för att säkerhetsställa funktionaliteten i appen så har suttit och skrivit lite testfall och fått det att fungera. Läste även lite i ett dockument som beskriver tankarna hos de nordiska länderna gällande framtiden inom Personal Connected health and care.\\
	Proxyproblemet ställer fortfarande till det lite för mig. Det verkar som om jag måste vara online på intranätet om jag skall köra Visual Studio med Xamarin då Xamarins utvecklingsmiljö ligger på en server som man måste komma åt. Brandväggen blockar Xamarin Android Player om jag inte sitter på intranätet.
\end{diary}

\begin{diary}{Fredag}{2016-02-19 08:00-17:00}
	Började så smått skriva i utvecklingsmiljön för att få ett humm om hur det funkar med android. Bestämde mig för att det kommer bli android till slut. Fördelarna med en egen service för bluetooth gör att det blir bra.
\end{diary}
\newpage

\begin{diary}{Måndag}{2016-02-22 08:00-15:00}
\vecka{8}
	Började implementera ett skal för en android app, letade efter lösningar för att kunna testa bluetooth men det verkar som om jag måste ha en fysisk telefon. Fanns ett sätt att testa men då var jag tvungen att köra en brygga genom en viruell maskin som kör en androidenhets operativsystem. Lättare att testa fysiskt på en riktig enhet.
\end{diary}

\begin{diary}{Tisdag}{2016-02-23 08:00-16:00}
	Implementerade mycket i skalet för appen. Läste på om hur man programmerar android. Var mycket länge sedan jag gjorde det men det kommer tillbaka mer och mer. Kan hända att jag måste lägga mer tid på att lära mig android än vad jag först hade beräknat. Tiden vill ge svar på det. Skrev även lite på rapporten som jag börjar bli mer och mer stressad över. Kommer sitta med rapporten hemma ikväll verkar det som då jag kommer förlora en halv dag på torsdag i och med MSB.
\end{diary}

\begin{diary}{Onsdag}{2016-02-24 08:00-16:00}
	Rapportskrivning
\end{diary}

\begin{diary}{Torsdag}{2016-02-25}
	MSB
\end{diary}

\begin{diary}{Fredag}{2016-02-26 08:00-12:00}
	Rapportskrivning sedan handledarmöte.
\end{diary}

\begin{diary}{Måndag}{2016-02-29}
\vecka{9}
	Sjuk...
\end{diary}

\begin{diary}{Tisdag}{2016-03-01 08:00-18:00}
	Börjat med designen av appen, dåligt med ideer så har börjat implementera Antidote men får det inte att fungera.
\end{diary}

\begin{diary}{Onsdag}{2016-03-02 08:00-16:00}
	Försökt få Antidote att fungera, börjat med en utvecklingsmiljö i ubuntu för att se om det hjälper. Alla byggkommandon är i linux så kan vara det som är problemet. Vill helst inte använda ubuntu då allt annat fungerar i windows. Sitter nån timme med att försöka få NDK att integrera med Android Studio.
\end{diary}

\begin{diary}{Torsdag}{2016-03-03 08:00-17:00}
	Kämpade med att få ordning på NDK-build. Kommer behöva bygga Antidote från kommandotolken verkar det som.
\end{diary}

\begin{diary}{Fredag}{2016-03-04 08:00-17:00}
	Klarade av att bygga antidote men har nu problem med att länka in biblioteket. Appen startar men hittar inte biblioteket.
\end{diary}
\newpage

\begin{diary}{Måndag}{2016-03-07 08:00-17:00}
\vecka{10}
	För att få lite annat att tänka på och för att lättare kunna förstå vad som händer i Evothings appen samt olika python filer jag har stött på så har jag bestämt mig för att läsa in javascript och python syntax. Gick igenom lite olika moduler på codeacademy och lärde mig grunderna i dessa språk.
\end{diary}

\begin{diary}{Tisdag}{2016-03-08 08:00-17:00}
	Fortsatta litteraturstudier i javascript och python.
\end{diary}

\begin{diary}{Onsdag}{2016-03-09 08:00-19:00}
	Fortsatta litteraturstudier i javascript och python.
\end{diary}

\begin{diary}{Torsdag}{2016-03-10 08:00-17:00}
	Fortsatta litteraturstudier i javascript och python. Klar nu och ska testa en sista gång med att få biblioteket jag behöver för appen att länka innan jag ger upp med det och kör med Evothing.
\end{diary}

\begin{diary}{Fredag}{2016-03-11}
	Sjuk. Handledarmöte
\end{diary}

\begin{diary}{Måndag}{2016-03-14}
\vecka{11}
	Sjuk.
\end{diary}

\begin{diary}{Tisdag}{2016-03-15 08:00-17:00}
	Fick äntligen biblioteket att länka idag. Ett nytt problem uppstod dock; biblioteket hittar inte ett annat bibliotek som det är beroende av. Ett steg närmare i alla fall.
\end{diary}

\begin{diary}{Onsdag}{2016-03-16 08:00-17:00}
	Efter att ha fått biblioteket att länka över huvud taget så var det inga problem att hitta en lösning för hur man länkar in beroende-biblioteket. Det var bara att lägga in en liten fix i källkoden där man säger åt appen all ladda in det andra biblioteket. Nu kan man se att det kommer in data till telefonen från mätutrustningen och jag har suttit och försökt förstå vad som egentligen händer.
\end{diary}

\begin{diary}{Torsdag}{2016-03-17 08:00-17:00}
	Satt och exprementerade hela dagen och försökte förstå hur det funkade samt försökte få en bluetooth-enhet som ej är Continua certifierad men det går inte. Ser dock hur de olika bitarna hänger ihop mer och mer. Trixar med koden för att få till så att appen inte skriver över senaste värdet vid fler än en inläsning samtidigt(blodtryck/puls).
\end{diary}

\begin{diary}{Fredag}{2016-03-18 08:00-15:00}
	Appen fungerar bra nu och jag har haft en demonstration på sprintmötet i morse. Jag har nu börjat med att se hur eSense tar emot data från brukaren och bestämde mig för att spara datan på samma format i appen så det lätt ska gå att skicka vidare den sedan. Har idag suttit med firstobject och studerat formatet som appen tar emot datan med. Är ett xml-format men det krånglar lite med blodtrycket och appen tappar bort enheten på formatet.
\end{diary}
\newpage

\begin{diary}{Måndag}{2016-03-21 08:00-17:00}
\vecka{12}
	Bestämde mig för att skrota sättet som Antidote gruppen läser in datan från HealthService helt. Det kommer bli lättare att skriva en egen parser så man får de värden man vill ha. Det var dess utom en del problem med hur datan presenterades för användaren och den tappade bort senaste mätningen.
\end{diary}

\begin{diary}{Tisdag}{2016-03-22 08:00-21:00}
	Mycket roligt att programmera nu när det fungerar och jag kan få se mätvärden i telefonen. Jag har säkert gjort 20 mätningar idag och gjort finjusteringar. Stannade länge för att jag skulle kunna få ordning på parsningen av xml till json men klarade inte gå hela vägen.
\end{diary}

\begin{diary}{Onsdag}{2016-03-23 08:00-17:00}
	Fortsatte med att få till parsningen men jag blir bara förvirrad. Hade varit så mycket lättare om den skickade en mätning med blodtrycken och en med pulsen och inte en med båda. Fick till slut till det så nu parsar den utan problem och jag bygger ett JsonObject som ser ut som den indata eSense tar emot. Har även fått till det så att jag genererar UUID och gör en tolkning av timestamp så användaren kan se tiden i klartext.
\end{diary}

\begin{diary}{Torsdag}{2016-03-24}
	Skärtorsdag. Tieto har stängt och jag kommer inte in. Passar på att vila.
\end{diary}

\begin{diary}{Fredag}{2016-03-25}
	Långfredag. Åker till Göteborg men tar inte med mig datorn. Kommer inte få något gjort i vilket fall.
\end{diary}

\begin{diary}{Måndag}{2016-03-28}
\vecka{13}
	Annandag påsk. Kommer hem från Göteborg. Inget gjort.
\end{diary}

\begin{diary}{Tisdag}{2016-03-29 08:00-21:00}
	Skapar en helt ny app för att få till det med mer kontroll på utseendet. Mycket googlande på hur man startar nya aktiviteter. Får till det men krånglar vidare med UI.
\end{diary}

\begin{diary}{Onsdag}{2016-03-30 08:00-22:00}
	Mockar data för att slippa göra mätningar på mig själv hela tiden. Fortsätter med en utvecklingsmiljö i den nya appen.
\end{diary}

\begin{diary}{Torsdag}{2016-03-31}
	Sjuk.
\end{diary}

\begin{diary}{Fredag}{2016-04-01}
	Sjuk.
\end{diary}
\newpage

\begin{diary}{Måndag}{2016-04-04 08:00-17:00}
\vecka{14}
	Börjar på att kolla på uppkopplingen till Meteor, en servertjänst eSense använder sig av för att vidarebefodra data till Home API. Det finns bibliotek  med exempel till android som har utvecklas för att bygga native apps. Implementerar detta och ser om jag lyckas.
\end{diary}

\begin{diary}{Tisdag}{2016-04-05 08:00-17:00}
	Jobbar vidare med att få androidklienten till meteor att fungera. Märkte att nätverkstrafiken skickade ett svagt ping men den gav inget utslag. Det är som att den kopplar ned direkt efter den kopplat upp. Hålls servicen vid liv?
\end{diary}

\begin{diary}{Onsdag}{2016-04-06 08:00-17:00}
	Ger tillfälligt upp på att köra service i bakgrunden och gör en lösning där jag håller aktiviteten vid liv. Kan nu koppla upp mot serven, skicka data och läsa av callbacks genom olika manuella kommandon. Kommer bli tvungen att anpassa detta till en bakgrundsservice. Läser på om skillnaden mellan Service och IntentService.
\end{diary}

\begin{diary}{Torsdag}{2016-04-07 08:00-17:00}
	Båda delarna av projektet fungerar nu men separat. Jag kan ta emot data från device och skicka till meteor men inte i samma app. Kommer lägga tid framöver på att lägga till funktionaliteten i HealthServiceTest till den i MobileGateway. Sändfunktionen ser dock fortfarande i en aktiv activity.
\end{diary}

\begin{diary}{Fredag}{2016-04-08 08:00-17:00}
	Handledarmöte. Satt i skolan och skrev på progressrapporten.
\end{diary}

\begin{diary}{Måndag}{2016-04-11 08:00-17:00}
\vecka{15}
	Försöker sätta över funktionaliteten vidare. Konstiga fel gör att appen krashar och jag får inga felmeddelanden då den krashar i native biblioteket. Ser inte vad som ska göras för att få det att fungera.
\end{diary}

\begin{diary}{Tisdag}{2016-04-12 08:00-17:00}
	Får äntligen ordning på inDatan som kommer från mätutrustningen. Största problemet var att jag inte kunde se vart som problemet låg. Det visade sig att jaag hade glömt att initiera en lista som jag sedan försökte lägga till object i. Då felmeddelandet kom från Antidote var detta svårt att upptäcka och jag fick lägga till en del i taget tills det till slut fungerade.
\end{diary}

\begin{diary}{Onsdag}{2016-04-13 08:00-17:00}
	Jobbar vidare med att få en service att köra i bakgrunden. 
\end{diary}

\begin{diary}{Torsdag}{2016-04-14 08:00-17:00}
	Lyckas äntligen få servicen för kommunikation att köra i bakgrunden. Tar små steg och ser till att Servicen kör medans jag lägger till implementationen för MeteorCallback. Klarar av att göra det som activity-versonen av kommunikations klassen kan vid slutet av dagen.
\end{diary}

\begin{diary}{Fredag}{2016-04-15}
	Hemma idag.
\end{diary}
\newpage

\begin{diary}{Måndag}{2016-04-18 08:00-17:00}
\vecka{16}
	Idag byggde jag om inne på servern för att lägga till min metod för att ta emot data. Krånglade en massa med hur data ska skickas med Object[] param formatet. Tog nästan hela dagen att få ordning på det. Lyckades skicka allt som behövs för att meteor ska kunna vidarebefodra förutom measurement stringen.
\end{diary}

\begin{diary}{Tisdag}{2016-04-19 08:00-17:00}
	Fick äntligen end to end kommunikationen att fungera. Det gick att parsa om ett json-sträng till json-object på servern i javascript vilket löste problemet från igår. Nu kan jag ta tag i rapportskrivandet tills jag har kommit ikapp med det. Rapportskrivande resten av dagen. Börjar med att uppdatera dagboken med händelser från de gångna veckorna.
\end{diary}

\begin{diary}{Onsdag}{2016-04-06 08:00-17:00}

\end{diary}

\begin{diary}{Torsdag}{2016-04-07 08:00-17:00}

\end{diary}

\begin{diary}{Fredag}{2016-04-08 08:00-17:00}

\end{diary}



\end{document}